We start by comparing how our prototype- and instance-based implementations of CMR account for behavior in a classic experiment where each item is presented just once per study phase. For these simulations, we used the dataset reported by \markdownRendererTextCite{1}+{}{}{murdock1970interresponse}. Each of 72 undergraduates performed 20 trials with study lists each consisting of 20 unique words visually presented at either 60 or 120 words per minute. Given a particular subject, words were unique both within and across trials, and randomly selected from the Toronto Word Pool \markdownRendererCite{1}+{}{}{friendly1982toronto}, a widely-used collection of high frequency nouns, adjectives, and verbs.\markdownRendererInterblockSeparator
{}While the major focus of the original report by \markdownRendererTextCite{1}+{}{}{murdock1970interresponse} was to investigate inter-response times in single-trial free recall, here we focus consideration on the content of recorded recall sequences. Because it excludes within-list repetitions of studied items, this dataset presents the opportunity to compare model performance under simplified conditions. Since items' feature representations are assumed orthogonal under considered variants of CMR, retrieving a pattern of contextual associations given an item-based cue only requires abstraction over the cued item's pre-experimental and single experimental contextual associations. Interpretation of apparent differences in performance across model variants thus focus primarily on mechanisms for context-based item representation retrieval.\markdownRendererInterblockSeparator
{}We compared the original prototype-based implementation of CMR against our two variants of our novel instance-based implementation. ICMR-Trace applies its nonlinear activation scaling parameter $\tau$ like traditional instance-based models, modulating the unique activation values assigned to each stored trace in memory before integration of an echo representation for retrieval. ICMR-Echo applies the parameter in a manner more like the traditional prototype-based specification of CMR, instead modulating item retrieval supports encoded in echo representations integrated over stored traces in memory. By considering all three model variants, the significance of this mechanistic distinction for model performance can be evaluated independently of other differences between instance-based and prototype-based model architectures.\markdownRendererInterblockSeparator
{}First we evaluated each model variant based on their ability to predict the specific sequences of recalls exhibited by each participant. Considering all 20 trials performed by each participant in the dataset, we applied the differential evolution optimization technique to find for each model the parameter configuration that maximized the likelihood of recorded recall sequences. We obtained a unique optimal parameter configuration for each unique participant and each considered model variant. To measure the goodness-of-fit for each parameter configuration and corresponding model, Figure ~\ref{fig:MurdOkaFits} plots the log-likelihood of each participant's recall sequences given each model variant's corresponding optimized parameter configuration. The median log-likelihood across participants for the CMR, ICMR-Echo, and ICMR-Trace model variants were found to be 299.5, 302.4, and 304.2, respectively, suggesting little meaningful difference between variants in their effectiveness accounting for participant recall performance across the dataset.\markdownRendererInterblockSeparator
{}As a follow-up, we also compared how readily each model could account for organizational summary statistics in the dataset. Focusing on prototype-based CMR and our main instance-based CMR implementation (ICMR-Trace), we found for each model the optimal parameter configuration maximizing the likelihood of the entire dataset rather than participant-by-participant. Using each fitted model variant, we simulated 1000 unique free recall trials and measured summary statistics from the result. Figure ~\ref{fig:MurdOkaSummary} plots for each model against the corresponding statistics collected over the dataset how recall probability varies as a function of serial position, how the probability of recalling an item first varies as a function of serial position, and how the conditional recall probabability of an item varies as a function of its serial lag from the previously recalled item. Recapitulating our comparison of log-likelihood distributions fitted over discrete participants, we found that both our prototype-based and instance-based CMR implementations account for these benchmark organizational summary statistics across the full dataset to similar extents. To build on this finding of broad model equivalence with respect to the results reported by \markdownRendererTextCite{1}+{}{}{murdock1970interresponse}, we next consider the model variants under other experimental conditions.\relax