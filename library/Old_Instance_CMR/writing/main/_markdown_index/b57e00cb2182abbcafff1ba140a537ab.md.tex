One clear conclusion we can draw from these analyses is that the success of the theoretical commitments made by the context maintenance and retrieval model and models like it are not especially dependent on the architectures in which they are implemented. Instead, the main insights of retrieved context theories (at least as formalized by CMR) are highly portable, and can likely be esconced within any reasonable model architecture where memory search via temporal contextual representations might prove valuable. This finding has been increasingly validated lately in other work. \markdownRendererCite{1}+{}{}{logan2021serial} for example similarly embeds mechanisms for maintaining and organizing retrieval using temporal contextual representations within an instance-based architecture to simultaneously account for performance on substantively distinct variations of a task requiring participants to encode and report random strings in left-to-right order by typing them on a computer keyboard, including whole report, serial recall, and copy typing. Other projects more motivated by neuroscientific data (e.g. \markdownRendererCite{1}+{}{}{ketz2013theta}; \markdownRendererCite{1}+{}{}{schapiro2017complementary}) embed mechanisms for context-based retrieval within detailed formal accounts of hippocampus functionality more complex than either the instance-based or hebbian associative network architectures considered in this work. The application of this instance-based account of context maintenance and retrieval to competitively explain free recall performance under varied conditions further clarify the architectural independence of the retrieved context account of memory search.\markdownRendererInterblockSeparator
{}How do these results fit into the context of other work contrasting instance- and prototype-based models? Research on category learning \markdownRendererCite{2}+{}{}{nosofsky2002exemplar}+{}{}{stanton2002comparisons} and semantic memory \markdownRendererTextCite{1}+{}{}{jamieson2018instance} emphasize that the main limitation of prototype-based models is the information that they discard or suppress at encoding - idiosyncratic item or contextual features that do not reflect generalities across experience. With this information discarded or suppressed, memory cues selective for those idiosyncratic features cannot result in retrieval of relevant information. We can conclude that either the considered research conditions or the model specifications themselves sidestep this issue. Two assumptions enforced in both the prototype- and instance-based frameworks compared here as well as in corresponding datasets were that list item were effectively representationally orthogonal, and encountered just once or twice before retrieval. Furthermore, contextual states as characterized by CMR differ a consistent amount from item to item during study in a traditional list learning experiment. The assumptions together may prevent a distinction from emerging between highly common and highly idiosyncratic item or contextual features under traditional research conditions.\markdownRendererInterblockSeparator
{}Higher rates of item repetition or enforced distortions of contextual variation (such as by dividing an encoding phase into distinct trials or sessions) might be enough to more clearly distinguish architecture performance. Simulations of high rates of item repetitions report in Figure ~\ref{fig:repeffect} identify one potentially relevant difference between InstanceCMR and PrototypeCMR -- an exponential rate of increase of recall rates for repeated items in the former, but not the latter -- but the distinction seems independent from contrasts drawn between the architectures drawn by other researchers such as \markdownRendererTextCite{1}+{}{}{jamieson2018instance} and \markdownRendererTextCite{1}+{}{}{nosofsky2002exemplar}. By contrast, \markdownRendererInterblockSeparator
{}These results also help advance recent efforts to develop "a general account of memory and its processes in a working computational system to produce a common explanation of behavior rather than a set of lab-specific and domain-specific theories for different behaviors" (\markdownRendererCite{1}+{}{}{jamieson2018instance} citing \markdownRendererTextCite{1}+{}{}{newell1973you}). As I review above,\relax