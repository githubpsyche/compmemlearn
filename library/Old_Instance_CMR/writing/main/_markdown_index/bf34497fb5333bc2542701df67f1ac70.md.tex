Retrieved context theories explain memory search in terms of interactions between between two representations across experience: one of temporal context (a context layer, $C$) and another of features of studied items (an item layer, $F$). While this paper introduces an instance-based account of these interactions, we specify a variant of the original prototype-based context maintenance and retrieval (CMR model) \markdownRendererCite{1}+{}{}{polyn2009context} to support comparison against this account. The instance-based model we emphasize tracks the history of interactions between context and item features by storing a discrete record of each experience in memory for later inspection. In contrast, prototypical CMR maintains a simplified neural network whose connection weights accumulate a collapsed average representation of context and item interactions across experience. Table ~\ref{table:cmr_parameters} provides an overview of the parameters and structures that determine model behavior.\markdownRendererInterblockSeparator
{}\begin{longtable}[]{@{}llll@{}} \toprule Structure Type & Symbol & Name & Description \ \midrule \endhead \multirow{3}{8em}{Architecture} \ & $C$ & temporal context & A recency-weighted average of encoded items \ & $F$ & item features & Current pattern of item feature unit activations \ & $M^{FC}$ & & encoded feature-to-context associations \ & $M^{CF}$ & & encoded context-to-feature associations \ \multirow{3}{8em}{Context Updating} \ & ${\beta}\markdownRendererEmphasis{{enc}$ & encoding drift rate & Rate of context drift during item encoding \ & ${\beta}}{start}$ & start drift rate & Amount of start-list context retrieved at start of recall \ & ${\beta}\markdownRendererEmphasis{{rec}$ & recall drift rate & Rate of context drift during recall \ \multirow{5}{8em}{Associative Structure} \ & ${\alpha}$ & shared support & Amount of support items initially have for one another \ & ${\delta}$ & item support & Initial pre-experimental contextual self-associations \ & ${\gamma}$ & learning rate & Amount of experimental context retrieved by a recalled item \ & ${\phi}}{s}$ & primacy scale & Scaling of primacy gradient on trace activations \ & ${\phi}\markdownRendererEmphasis{{d}$ & primacy decay & Rate of decay of primacy gradient \ \multirow{3}{8em}{Retrieval Dynamics} \ & ${\tau}$ & choice sensitivity & Exponential weighting of similarity-driven activation \ & ${\theta}}{s}$ & stop probability scale & Scaling of the stop probability over output position \ & ${\theta}\markdownRendererEmphasis{{r}$ & stop probability growth & Rate of increase in stop probability over output position \ \bottomrule \ \caption{Parameters and structures specifying CMR} \label{table:cmr}parameters} \end{longtable}\markdownRendererInterblockSeparator
{}\markdownRendererHeadingTwo{Initial State}\markdownRendererInterblockSeparator
{}Associative connections built within this network are represented by matrices $M^{FC}$ and $M^{CF}$.\markdownRendererInterblockSeparator
{}To summarize pre-experimental associations built between relevant item features and possible contextual states, we initialize $M^{FC}$ accordingly\markdownRendererInterblockSeparator
{}$$M^{FC}_{pre(ij)} = \begin{cases} \begin{alignedat}{2} 1 - \gamma \text{, if } i=j \ 0 \text{, if } i \neq j\ \end{alignedat} \end{cases}$$\markdownRendererInterblockSeparator
{}This connects each unit on $F$ to a unique unit on $C$. Used this way, $\gamma$ controls the relative contribution of pre-experimentally acquired associations to the course of retrieval compared to experimentally acquired associations. Correspondingly, context-to-feature associations tracked by $M^{CF}$ are set according to:\markdownRendererInterblockSeparator
{}$$M^{CF}_{pre(ij)} = \begin{cases} \begin{alignedat}{2} 1 - \delta \text{, if } i=j \ \alpha \text{, if } i \neq j\ \end{alignedat} \end{cases}$$\markdownRendererInterblockSeparator
{}Like $\gamma$ with respect to $M^{FC}$, the $\delta$ parameter controls the contribution of pre-experimental context-to-feature associations relative to experimentally acquired ones. Since context-to-feature associations organizes the competition of items for retrieval, the $\alpha$ parameter further causes items to support one another in retrieval in a uniform way.\markdownRendererInterblockSeparator
{}Context is initialized with a state orthogonal to any of those pre-experimentally associated with an relevant item feature. Feature representations corresponding to items are also assumed to be orthonormal with respect to one another such that each unit on $F$ corresponds to one item.\relax