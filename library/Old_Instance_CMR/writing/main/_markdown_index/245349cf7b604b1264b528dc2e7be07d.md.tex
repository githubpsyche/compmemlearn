Retrieved context theories explain memory search in terms of interactions between between two representations across experience: one of temporal context (a context layer, \markdownRendererDollarSign{}C\markdownRendererDollarSign{}) and another of features of studied items (an item layer, \markdownRendererDollarSign{}F\markdownRendererDollarSign{}). While this paper introduces an instance-based account of these interactions, we specify a variant of the original prototype-based context maintenance and retrieval (CMR model) [@polyn2009context] to support comparison against this account. The instance-based model we emphasize tracks the history of interactions between context and item features by storing a discrete record of each experience in memory for later inspection. In contrast, prototypical CMR maintains a simplified neural network whose connection weights accumulate a collapsed average representation of context and item interactions across experience. Table  \markdownRendererLink{1}{\markdownRendererHash{}table:cmr\markdownRendererUnderscore{}parameters}{#table:cmr_parameters}{}\markdownRendererLeftBrace{}reference-type="ref" reference="table:cmr\markdownRendererUnderscore{}parameters"\markdownRendererRightBrace{} provides an overview of the parameters and structures that determine model behavior.\markdownRendererInterblockSeparator
{}::: \markdownRendererLeftBrace{}\markdownRendererHash{}table:cmr\markdownRendererEmphasis{parameters\markdownRendererRightBrace{} Structure Type Symbol Name Description ----------------------- ------------------- ------------------------- ------------------------------------------------------------- Architecture\markdownRendererLineBreak
{}\markdownRendererDollarSign{}C\markdownRendererDollarSign{} temporal context A recency-weighted average of encoded items \markdownRendererDollarSign{}F\markdownRendererDollarSign{} item features Current pattern of item feature unit activations \markdownRendererDollarSign{}M\markdownRendererCircumflex{}\markdownRendererLeftBrace{}FC\markdownRendererRightBrace{}\markdownRendererDollarSign{} encoded feature-to-context associations \markdownRendererDollarSign{}M\markdownRendererCircumflex{}\markdownRendererLeftBrace{}CF\markdownRendererRightBrace{}\markdownRendererDollarSign{} encoded context-to-feature associations Context Updating\markdownRendererLineBreak
{}\markdownRendererDollarSign{}\markdownRendererLeftBrace{}\markdownRendererBackslash{}beta\markdownRendererRightBrace{}}\markdownRendererLeftBrace{}enc\markdownRendererRightBrace{}\markdownRendererDollarSign{} encoding drift rate Rate of context drift during item encoding \markdownRendererDollarSign{}\markdownRendererLeftBrace{}\markdownRendererBackslash{}beta\markdownRendererRightBrace{}\markdownRendererEmphasis{\markdownRendererLeftBrace{}start\markdownRendererRightBrace{}\markdownRendererDollarSign{} start drift rate Amount of start-list context retrieved at start of recall \markdownRendererDollarSign{}\markdownRendererLeftBrace{}\markdownRendererBackslash{}beta\markdownRendererRightBrace{}}\markdownRendererLeftBrace{}rec\markdownRendererRightBrace{}\markdownRendererDollarSign{} recall drift rate Rate of context drift during recall Associative Structure\markdownRendererLineBreak
{}\markdownRendererDollarSign{}\markdownRendererLeftBrace{}\markdownRendererBackslash{}alpha\markdownRendererRightBrace{}\markdownRendererDollarSign{} shared support Amount of support items initially have for one another \markdownRendererDollarSign{}\markdownRendererLeftBrace{}\markdownRendererBackslash{}delta\markdownRendererRightBrace{}\markdownRendererDollarSign{} item support Initial pre-experimental contextual self-associations \markdownRendererDollarSign{}\markdownRendererLeftBrace{}\markdownRendererBackslash{}gamma\markdownRendererRightBrace{}\markdownRendererDollarSign{} learning rate Amount of experimental context retrieved by a recalled item \markdownRendererDollarSign{}\markdownRendererLeftBrace{}\markdownRendererBackslash{}phi\markdownRendererRightBrace{}\markdownRendererEmphasis{\markdownRendererLeftBrace{}s\markdownRendererRightBrace{}\markdownRendererDollarSign{} primacy scale Scaling of primacy gradient on trace activations \markdownRendererDollarSign{}\markdownRendererLeftBrace{}\markdownRendererBackslash{}phi\markdownRendererRightBrace{}}\markdownRendererLeftBrace{}d\markdownRendererRightBrace{}\markdownRendererDollarSign{} primacy decay Rate of decay of primacy gradient Retrieval Dynamics\markdownRendererLineBreak
{}\markdownRendererDollarSign{}\markdownRendererLeftBrace{}\markdownRendererBackslash{}tau\markdownRendererRightBrace{}\markdownRendererDollarSign{} choice sensitivity Exponential weighting of similarity-driven activation \markdownRendererDollarSign{}\markdownRendererLeftBrace{}\markdownRendererBackslash{}theta\markdownRendererRightBrace{}\markdownRendererEmphasis{\markdownRendererLeftBrace{}s\markdownRendererRightBrace{}\markdownRendererDollarSign{} stop probability scale Scaling of the stop probability over output position \markdownRendererDollarSign{}\markdownRendererLeftBrace{}\markdownRendererBackslash{}theta\markdownRendererRightBrace{}}\markdownRendererLeftBrace{}r\markdownRendererRightBrace{}\markdownRendererDollarSign{} stop probability growth Rate of increase in stop probability over output position\markdownRendererInterblockSeparator
{}: Parameters and structures specifying CMR :::\markdownRendererInterblockSeparator
{}\markdownRendererHeadingTwo{Initial State}\markdownRendererInterblockSeparator
{}Associative connections built within this network are represented by matrices \markdownRendererDollarSign{}M\markdownRendererCircumflex{}\markdownRendererLeftBrace{}FC\markdownRendererRightBrace{}\markdownRendererDollarSign{} and \markdownRendererDollarSign{}M\markdownRendererCircumflex{}\markdownRendererLeftBrace{}CF\markdownRendererRightBrace{}\markdownRendererDollarSign{}.\markdownRendererInterblockSeparator
{}To summarize pre-experimental associations built between relevant item features and possible contextual states, we initialize \markdownRendererDollarSign{}M\markdownRendererCircumflex{}\markdownRendererLeftBrace{}FC\markdownRendererRightBrace{}\markdownRendererDollarSign{} accordingly\markdownRendererInterblockSeparator
{}\markdownRendererDollarSign{}\markdownRendererDollarSign{}M\markdownRendererCircumflex{}\markdownRendererLeftBrace{}FC\markdownRendererRightBrace{}\markdownRendererUnderscore{}\markdownRendererLeftBrace{}pre(ij)\markdownRendererRightBrace{} = \markdownRendererBackslash{}begin\markdownRendererLeftBrace{}cases\markdownRendererRightBrace{} \markdownRendererBackslash{}begin\markdownRendererLeftBrace{}alignedat\markdownRendererRightBrace{}\markdownRendererLeftBrace{}2\markdownRendererRightBrace{} 1 - \markdownRendererBackslash{}gamma \markdownRendererBackslash{}text\markdownRendererLeftBrace{}, if \markdownRendererRightBrace{} i=j \markdownRendererBackslash{} 0 \markdownRendererBackslash{}text\markdownRendererLeftBrace{}, if \markdownRendererRightBrace{} i \markdownRendererBackslash{}neq j\markdownRendererBackslash{} \markdownRendererBackslash{}end\markdownRendererLeftBrace{}alignedat\markdownRendererRightBrace{} \markdownRendererBackslash{}end\markdownRendererLeftBrace{}cases\markdownRendererRightBrace{}\markdownRendererDollarSign{}\markdownRendererDollarSign{}\markdownRendererInterblockSeparator
{}This connects each unit on \markdownRendererDollarSign{}F\markdownRendererDollarSign{} to a unique unit on \markdownRendererDollarSign{}C\markdownRendererDollarSign{}. Used this way, \markdownRendererDollarSign{}\markdownRendererBackslash{}gamma\markdownRendererDollarSign{} controls the relative contribution of pre-experimentally acquired associations to the course of retrieval compared to experimentally acquired associations. Correspondingly, context-to-feature associations tracked by \markdownRendererDollarSign{}M\markdownRendererCircumflex{}\markdownRendererLeftBrace{}CF\markdownRendererRightBrace{}\markdownRendererDollarSign{} are set according to:\markdownRendererInterblockSeparator
{}\markdownRendererDollarSign{}\markdownRendererDollarSign{}M\markdownRendererCircumflex{}\markdownRendererLeftBrace{}CF\markdownRendererRightBrace{}\markdownRendererUnderscore{}\markdownRendererLeftBrace{}pre(ij)\markdownRendererRightBrace{} = \markdownRendererBackslash{}begin\markdownRendererLeftBrace{}cases\markdownRendererRightBrace{} \markdownRendererBackslash{}begin\markdownRendererLeftBrace{}alignedat\markdownRendererRightBrace{}\markdownRendererLeftBrace{}2\markdownRendererRightBrace{} 1 - \markdownRendererBackslash{}delta \markdownRendererBackslash{}text\markdownRendererLeftBrace{}, if \markdownRendererRightBrace{} i=j \markdownRendererBackslash{} \markdownRendererBackslash{}alpha \markdownRendererBackslash{}text\markdownRendererLeftBrace{}, if \markdownRendererRightBrace{} i \markdownRendererBackslash{}neq j\markdownRendererBackslash{} \markdownRendererBackslash{}end\markdownRendererLeftBrace{}alignedat\markdownRendererRightBrace{} \markdownRendererBackslash{}end\markdownRendererLeftBrace{}cases\markdownRendererRightBrace{}\markdownRendererDollarSign{}\markdownRendererDollarSign{}\markdownRendererInterblockSeparator
{}Like \markdownRendererDollarSign{}\markdownRendererBackslash{}gamma\markdownRendererDollarSign{} with respect to \markdownRendererDollarSign{}M\markdownRendererCircumflex{}\markdownRendererLeftBrace{}FC\markdownRendererRightBrace{}\markdownRendererDollarSign{}, the \markdownRendererDollarSign{}\markdownRendererBackslash{}delta\markdownRendererDollarSign{} parameter controls the contribution of pre-experimental context-to-feature associations relative to experimentally acquired ones. Since context-to-feature associations organizes the competition of items for retrieval, the \markdownRendererDollarSign{}\markdownRendererBackslash{}alpha\markdownRendererDollarSign{} parameter further causes items to support one another in retrieval in a uniform way.\markdownRendererInterblockSeparator
{}Context is initialized with a state orthogonal to any of those pre-experimentally associated with an relevant item feature. Feature representations corresponding to items are also assumed to be orthonormal with respect to one another such that each unit on \markdownRendererDollarSign{}F\markdownRendererDollarSign{} corresponds to one item.\markdownRendererInterblockSeparator
{}\markdownRendererHeadingTwo{Encoding Phase}\markdownRendererInterblockSeparator
{}Whenever an item \markdownRendererDollarSign{}i\markdownRendererDollarSign{} is presented for study, its corresponding feature representation \markdownRendererDollarSign{}f\markdownRendererUnderscore{}i\markdownRendererDollarSign{} is activated on \markdownRendererDollarSign{}F\markdownRendererDollarSign{} and its contextual associations encoded into \markdownRendererDollarSign{}M\markdownRendererCircumflex{}\markdownRendererLeftBrace{}FC\markdownRendererRightBrace{}\markdownRendererDollarSign{} are retrieved, altering the current state of context \markdownRendererDollarSign{}C\markdownRendererDollarSign{}.\markdownRendererInterblockSeparator
{}The input to context is determined by:\markdownRendererInterblockSeparator
{}\markdownRendererDollarSign{}\markdownRendererDollarSign{}c\markdownRendererCircumflex{}\markdownRendererLeftBrace{}IN\markdownRendererRightBrace{}\markdownRendererEmphasis{\markdownRendererLeftBrace{}i\markdownRendererRightBrace{} = M\markdownRendererCircumflex{}\markdownRendererLeftBrace{}FC\markdownRendererRightBrace{}f}\markdownRendererLeftBrace{}i\markdownRendererRightBrace{}\markdownRendererDollarSign{}\markdownRendererDollarSign{}\markdownRendererInterblockSeparator
{}and normalized to have length 1. Context is updated based on this input according to:\markdownRendererInterblockSeparator
{}\markdownRendererDollarSign{}\markdownRendererDollarSign{}c\markdownRendererEmphasis{i = \markdownRendererBackslash{}rho}ic\markdownRendererEmphasis{\markdownRendererLeftBrace{}i-1\markdownRendererRightBrace{} + \markdownRendererBackslash{}beta}\markdownRendererLeftBrace{}enc\markdownRendererRightBrace{} c\markdownRendererUnderscore{}\markdownRendererLeftBrace{}i\markdownRendererRightBrace{}\markdownRendererCircumflex{}\markdownRendererLeftBrace{}IN\markdownRendererRightBrace{}\markdownRendererDollarSign{}\markdownRendererDollarSign{}\markdownRendererInterblockSeparator
{}with \markdownRendererDollarSign{}\markdownRendererBackslash{}beta\markdownRendererDollarSign{} (for encoding we use \markdownRendererDollarSign{}\markdownRendererBackslash{}beta\markdownRendererEmphasis{enc\markdownRendererDollarSign{}) shaping the rate of contextual drift with each new experience, and \markdownRendererDollarSign{}\markdownRendererBackslash{}rho\markdownRendererDollarSign{} enforces the length of \markdownRendererDollarSign{}c}i\markdownRendererDollarSign{} to 1 according to:\markdownRendererInterblockSeparator
{}\markdownRendererDollarSign{}\markdownRendererDollarSign{}\markdownRendererBackslash{}rho\markdownRendererEmphasis{i = \markdownRendererBackslash{}sqrt\markdownRendererLeftBrace{}1 + \markdownRendererBackslash{}beta\markdownRendererCircumflex{}2\markdownRendererBackslash{}left[\markdownRendererBackslash{}left(c\markdownRendererEmphasis{\markdownRendererLeftBrace{}i-1\markdownRendererRightBrace{} \markdownRendererBackslash{}cdot c\markdownRendererCircumflex{}\markdownRendererLeftBrace{}IN\markdownRendererRightBrace{}}i\markdownRendererBackslash{}right)\markdownRendererCircumflex{}2 - 1\markdownRendererBackslash{}right]\markdownRendererRightBrace{} - \markdownRendererBackslash{}beta\markdownRendererBackslash{}left(c}\markdownRendererLeftBrace{}i-1\markdownRendererRightBrace{} \markdownRendererBackslash{}cdot c\markdownRendererCircumflex{}\markdownRendererLeftBrace{}IN\markdownRendererRightBrace{}\markdownRendererUnderscore{}i\markdownRendererBackslash{}right)\markdownRendererDollarSign{}\markdownRendererDollarSign{}\markdownRendererInterblockSeparator
{}Associations between each \markdownRendererDollarSign{}c\markdownRendererEmphasis{i\markdownRendererDollarSign{} and \markdownRendererDollarSign{}f}i\markdownRendererDollarSign{} are built through Hebbian learning:\markdownRendererInterblockSeparator
{}\markdownRendererDollarSign{}\markdownRendererDollarSign{}\markdownRendererBackslash{}Delta M\markdownRendererCircumflex{}\markdownRendererLeftBrace{}FC\markdownRendererRightBrace{}\markdownRendererEmphasis{\markdownRendererLeftBrace{}exp\markdownRendererRightBrace{} = \markdownRendererBackslash{}gamma c}i f\markdownRendererCircumflex{}\markdownRendererLeftBrace{}'\markdownRendererRightBrace{}\markdownRendererUnderscore{}i\markdownRendererDollarSign{}\markdownRendererDollarSign{}\markdownRendererInterblockSeparator
{}and\markdownRendererInterblockSeparator
{}\markdownRendererDollarSign{}\markdownRendererDollarSign{}\markdownRendererBackslash{}Delta M\markdownRendererCircumflex{}\markdownRendererLeftBrace{}CF\markdownRendererRightBrace{}\markdownRendererEmphasis{\markdownRendererLeftBrace{}exp\markdownRendererRightBrace{} = \markdownRendererBackslash{}phi}i f\markdownRendererEmphasis{i c\markdownRendererCircumflex{}\markdownRendererLeftBrace{}'\markdownRendererRightBrace{}}i\markdownRendererDollarSign{}\markdownRendererDollarSign{}\markdownRendererInterblockSeparator
{}where \markdownRendererDollarSign{}\markdownRendererBackslash{}phi\markdownRendererUnderscore{}i\markdownRendererDollarSign{} enforces a primacy effect, scales the amount of learning based on the serial position of the studied item according to\markdownRendererInterblockSeparator
{}\markdownRendererDollarSign{}\markdownRendererDollarSign{}\markdownRendererBackslash{}phi\markdownRendererEmphasis{i = \markdownRendererBackslash{}phi}se\markdownRendererCircumflex{}\markdownRendererLeftBrace{}-\markdownRendererBackslash{}phi\markdownRendererUnderscore{}d(i-1)\markdownRendererRightBrace{} + 1\markdownRendererDollarSign{}\markdownRendererDollarSign{}\markdownRendererInterblockSeparator
{}This function decays over time, such that \markdownRendererDollarSign{}\markdownRendererBackslash{}phi\markdownRendererEmphasis{se\markdownRendererDollarSign{} modulates the strength of primacy while \markdownRendererDollarSign{}\markdownRendererBackslash{}phi}d\markdownRendererDollarSign{} modulates the rate of decay.\markdownRendererInterblockSeparator
{}\markdownRendererHeadingTwo{Retrieval Phase}\markdownRendererInterblockSeparator
{}To help the model account for the primacy effect, we assume that between the encoding and retrieval phase of a task, the content of \markdownRendererDollarSign{}C\markdownRendererDollarSign{} has drifted some amoung back toward its pre-experimental state and set the state of context at the start of retrieval according to following, with \markdownRendererDollarSign{}\markdownRendererBackslash{}rho\markdownRendererDollarSign{} calculated as specified above:\markdownRendererInterblockSeparator
{}\markdownRendererDollarSign{}\markdownRendererDollarSign{}c\markdownRendererEmphasis{\markdownRendererLeftBrace{}start\markdownRendererRightBrace{} = \markdownRendererBackslash{}rho}\markdownRendererLeftBrace{}N+1\markdownRendererRightBrace{}c\markdownRendererEmphasis{N + \markdownRendererBackslash{}beta}\markdownRendererLeftBrace{}start\markdownRendererRightBrace{}c\markdownRendererUnderscore{}0\markdownRendererDollarSign{}\markdownRendererDollarSign{}\markdownRendererInterblockSeparator
{}At each recall attempt, the current state of context is used as a cue to attempt retrieval of some studied item. An activation \markdownRendererDollarSign{}a\markdownRendererDollarSign{} is solicited for each item according to:\markdownRendererInterblockSeparator
{}\markdownRendererDollarSign{}\markdownRendererDollarSign{}a = M\markdownRendererCircumflex{}\markdownRendererLeftBrace{}CF\markdownRendererRightBrace{}c\markdownRendererDollarSign{}\markdownRendererDollarSign{}\markdownRendererInterblockSeparator
{}Each item gets a minimum activation of \markdownRendererDollarSign{}10\markdownRendererCircumflex{}\markdownRendererLeftBrace{}-7\markdownRendererRightBrace{}\markdownRendererDollarSign{}. To determine the probability of a given recall event, we first calculate the probability of stopping recall - returning no item and ending memory search. This probability varies as a function of output position \markdownRendererDollarSign{}j\markdownRendererDollarSign{}:\markdownRendererInterblockSeparator
{}\markdownRendererDollarSign{}\markdownRendererDollarSign{}P(stop, j) = \markdownRendererBackslash{}theta\markdownRendererEmphasis{se\markdownRendererCircumflex{}\markdownRendererLeftBrace{}j\markdownRendererBackslash{}theta}r\markdownRendererRightBrace{}\markdownRendererDollarSign{}\markdownRendererDollarSign{}\markdownRendererInterblockSeparator
{}In this way, \markdownRendererDollarSign{}\markdownRendererBackslash{}theta\markdownRendererEmphasis{\markdownRendererLeftBrace{}se\markdownRendererRightBrace{}\markdownRendererDollarSign{} and \markdownRendererDollarSign{}theta}r\markdownRendererDollarSign{} control the scaling and rate of increase of this exponential function. Given that recall is not stopped, the probability \markdownRendererDollarSign{}P(i\markdownRendererDollarSign{}) of recalling a given item depends mainly on its activation strength according\markdownRendererInterblockSeparator
{}\markdownRendererDollarSign{}\markdownRendererDollarSign{}P(i) = (1-P(stop))\markdownRendererBackslash{}frac\markdownRendererLeftBrace{}a\markdownRendererCircumflex{}\markdownRendererLeftBrace{}\markdownRendererBackslash{}tau\markdownRendererRightBrace{}\markdownRendererEmphasis{i\markdownRendererRightBrace{}\markdownRendererLeftBrace{}\markdownRendererBackslash{}sum}\markdownRendererLeftBrace{}k\markdownRendererRightBrace{}\markdownRendererCircumflex{}\markdownRendererLeftBrace{}N\markdownRendererRightBrace{}a\markdownRendererCircumflex{}\markdownRendererLeftBrace{}\markdownRendererBackslash{}tau\markdownRendererRightBrace{}\markdownRendererUnderscore{}k\markdownRendererRightBrace{}\markdownRendererDollarSign{}\markdownRendererDollarSign{}\markdownRendererInterblockSeparator
{}\markdownRendererDollarSign{}\markdownRendererBackslash{}tau\markdownRendererDollarSign{} here shapes the contrast between well-supported and poorly supported items: exponentiating a large activation and a small activation by a large value of \markdownRendererDollarSign{}\markdownRendererBackslash{}tau\markdownRendererDollarSign{} widens the difference between those activations, making recall of the most activated item even more likely. Small values of \markdownRendererDollarSign{}\markdownRendererBackslash{}tau\markdownRendererDollarSign{} can alternatively driven recall likelihoods of differentially activated items toward one another.\markdownRendererInterblockSeparator
{}If an item is recalled, then that item is reactivated on \markdownRendererDollarSign{}F\markdownRendererDollarSign{}, and its contextual associations retrieved for integration into context again according to:\markdownRendererInterblockSeparator
{}\markdownRendererDollarSign{}\markdownRendererDollarSign{}c\markdownRendererCircumflex{}\markdownRendererLeftBrace{}IN\markdownRendererRightBrace{}\markdownRendererEmphasis{\markdownRendererLeftBrace{}i\markdownRendererRightBrace{} = M\markdownRendererCircumflex{}\markdownRendererLeftBrace{}FC\markdownRendererRightBrace{}f}\markdownRendererLeftBrace{}i\markdownRendererRightBrace{}\markdownRendererDollarSign{}\markdownRendererDollarSign{}\markdownRendererInterblockSeparator
{}Context is updated again based on this input (using \markdownRendererDollarSign{}\markdownRendererBackslash{}beta\markdownRendererEmphasis{rec\markdownRendererDollarSign{} instead of \markdownRendererDollarSign{}\markdownRendererBackslash{}beta}enc\markdownRendererDollarSign{}) and used to cue a successive recall attempt. This process continues until recall stops.\relax