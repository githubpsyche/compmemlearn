\markdownRendererHeadingOne{The Prototype-Based Retrieved Context Account of Free Recall}\markdownRendererInterblockSeparator
{}Retrieved context theories explain memory search in terms of interactions between between two representations across experience: one of temporal context (a context layer, $C$) and another of features of studied items (an item layer, $F$). While this paper introduces an instance-based account of these interactions, we here specify a variant of the original prototype-based context maintenance and retrieval (CMR) model \markdownRendererCite{1}+{}{}{polyn2009context} to support comparison against this account. The instance-based model we emphasize tracks the history of interactions between context and item features by storing a discrete record of each experience in memory for later inspection. In contrast, prototypeCMR maintains a simplified neural network whose connection weights accumulate a center of tendency representation reflecting context and item interactions across experience. \markdownRendererInterblockSeparator
{}\markdownRendererTable{Parameters and structures specifying CMR}{20}{4}{llll}%
{{Structure Type}%
{Symbol}%
{Name}%
{Description}%
}%
{{Architecture}%
{}%
{}%
{}%
}%
{{}%
{$C$}%
{temporal context}%
{A recency-weighted average of encoded items}%
}%
{{}%
{$F$}%
{item features}%
{Current pattern of item feature unit activations}%
}%
{{}%
{$M^{FC}$}%
{}%
{encoded feature-to-context associations}%
}%
{{}%
{$M^{CF}$}%
{}%
{encoded context-to-feature associations}%
}%
{{Context Updating}%
{}%
{}%
{}%
}%
{{}%
{${\beta}_{enc}$}%
{encoding drift rate}%
{Rate of context drift during item encoding}%
}%
{{}%
{${\beta}_{start}$}%
{start drift rate}%
{Amount of start-list context retrieved at start of recall}%
}%
{{}%
{${\beta}_{rec}$}%
{recall drift rate}%
{Rate of context drift during recall}%
}%
{{Associative Structure}%
{}%
{}%
{}%
}%
{{}%
{${\alpha}$}%
{shared support}%
{Amount of support items initially have for one another}%
}%
{{}%
{${\delta}$}%
{item support}%
{Initial pre-experimental contextual self-associations}%
}%
{{}%
{${\gamma}$}%
{learning rate}%
{Amount of experimental context retrieved by a recalled item}%
}%
{{}%
{${\phi}_{s}$}%
{primacy scale}%
{Scaling of primacy gradient on trace activations}%
}%
{{}%
{${\phi}_{d}$}%
{primacy decay}%
{Rate of decay of primacy gradient}%
}%
{{Retrieval Dynamics}%
{}%
{}%
{}%
}%
{{}%
{${\tau}$}%
{choice sensitivity}%
{Exponential weighting of similarity-driven activation}%
}%
{{}%
{${\theta}_{s}$}%
{stop probability scale}%
{Scaling of the stop probability over output position}%
}%
{{}%
{${\theta}_{r}$}%
{stop probability growth}%
{Rate of increase in stop probability over output position}%
}%
\markdownRendererInterblockSeparator
{}\markdownRendererHeadingTwo{Initial State}\markdownRendererInterblockSeparator
{}Associative connections built within prototypeCMR are represented by matrices $M^{FC}$ and $M^{CF}$.\markdownRendererInterblockSeparator
{}To summarize pre-experimental associations built between relevant item features and possible contextual states, we initialize $M^{FC}$ according to:\markdownRendererInterblockSeparator
{}\begin{equation} \label{eq:1} M^{FC}_{pre(ij)} = \begin{cases} \begin{alignedat}{2} 1 - \gamma \text{, if } i=j \\ 0 \text{, if } i \neq j \end{alignedat} \end{cases} \end{equation}\markdownRendererInterblockSeparator
{}This connects each unit on $F$ to a unique unit on $C$. Used this way, $\gamma$ controls the relative contribution of pre-experimentally acquired associations to the course of retrieval compared to experimentally acquired associations. Correspondingly, context-to-feature associations tracked by $M^{CF}$ are set according to:\markdownRendererInterblockSeparator
{}\begin{equation} \label{eq:2} M^{CF}_{pre(ij)} = \begin{cases} \begin{alignedat}{2} 1 - \delta \text{, if } i=j \\ \alpha \text{, if } i \neq j \end{alignedat} \end{cases} \end{equation}\markdownRendererInterblockSeparator
{}Like $\gamma$ with respect to $M^{FC}$, the $\delta$ parameter controls the contribution of pre-experimental context-to-feature associations relative to experimentally acquired ones. Since context-to-feature associations organizes the competition of items for retrieval, the $\alpha$ parameter specifies a uniform baseline extent to which items support one another in that competition.\markdownRendererInterblockSeparator
{}Context is initialized with a state orthogonal to any of those pre-experimentally associated with an relevant item feature. Feature representations corresponding to items are also assumed to be orthonormal with respect to one another such that each unit on $F$ corresponds to one item.\markdownRendererInterblockSeparator
{}\markdownRendererHeadingTwo{Encoding Phase}\markdownRendererInterblockSeparator
{}Whenever an item $i$ is presented for study, its corresponding feature representation $f_i$ is activated on $F$ and its contextual associations encoded into $M^{FC}$ are retrieved, altering the current state of context $C$.\markdownRendererInterblockSeparator
{}The input to context is determined by:\markdownRendererInterblockSeparator
{}\begin{equation} \label{eq:3} c^{IN}_{i} = M^{FC}f_{i} \end{equation}\markdownRendererInterblockSeparator
{}and normalized to have length 1. Context is updated based on this input according to:\markdownRendererInterblockSeparator
{}\begin{equation} \label{eq:4} c_i = \rho_ic_{i-1} + \beta_{enc} c_{i}^{IN} \end{equation}\markdownRendererInterblockSeparator
{}with $\beta$ (for encoding we use $\beta_{enc}$) shaping the rate of contextual drift with each new experience, and $\rho$ enforces the length of $c_i$ to 1 according to:\markdownRendererInterblockSeparator
{}\begin{equation} \label{eq:5} \rho_i = \sqrt{1 + \beta^2\left[\left(c_{i-1} \cdot c^{IN}_i\right)^2 - 1\right]} - \beta\left(c_{i-1} \cdot c^{IN}_i\right)\end{equation}\markdownRendererInterblockSeparator
{}Associations between each $c_i$ and $f_i$ are built through Hebbian learning:\markdownRendererInterblockSeparator
{}\begin{equation} \label{eq:6 }\Delta M^{FC}_{exp} = \gamma c_i f^{'}_i \end{equation}\markdownRendererInterblockSeparator
{}and\markdownRendererInterblockSeparator
{}\begin{equation} \label{eq:7} \Delta M^{CF}_{exp} = \phi_i f_i c^{'}_i \end{equation}\markdownRendererInterblockSeparator
{}where $\phi_i$ enforces a primacy effect, scales the amount of learning based on the serial position of the studied item according to\markdownRendererInterblockSeparator
{}\begin{equation} \label{eq:8} \phi_i = \phi_se^{-\phi_d(i-1)} + 1 \end{equation}\markdownRendererInterblockSeparator
{}This function decays over time, such that $\phi_{s}$ modulates the strength of primacy while $\phi_{d}$ modulates the rate of decay.\markdownRendererInterblockSeparator
{}\markdownRendererHeadingTwo{Retrieval Phase}\markdownRendererInterblockSeparator
{}To help the model account for the primacy effect, we assume that between the encoding and retrieval phase of a task, the content of $C$ has drifted some amoung back toward its pre-experimental state and set the state of context at the start of retrieval according to following, with $\rho$ calculated as specified above:\markdownRendererInterblockSeparator
{}\begin{equation} \label{eq:9} c_{start} = \rho_{N+1}c_N + \beta_{start}c_0 \end{equation}\markdownRendererInterblockSeparator
{}At each recall attempt, the current state of context is used as a cue to attempt retrieval of some studied item. An activation $a$ is solicited for each item according to:\markdownRendererInterblockSeparator
{}\begin{equation} \label{eq:10} a = M^{CF}c \end{equation}\markdownRendererInterblockSeparator
{}Each item gets a minimum activation of $10^{-7}$. To determine the probability of a given recall event, we first calculate the probability of stopping recall - returning no item and ending memory search. This probability varies as a function of output position $j$:\markdownRendererInterblockSeparator
{}\begin{equation} \label{eq:11} P(stop, j) = \theta_se^{j\theta_r} \end{equation}\markdownRendererInterblockSeparator
{}In this way, $\theta_s$ and $\theta_r$ control the scaling and rate of increase of this exponential function. Given that recall is not stopped, the probability $P(i)$ of recalling a given item depends mainly on its activation strength according\markdownRendererInterblockSeparator
{}\begin{equation} \label{eq:12} P(i) = (1-P(stop))\frac{a^{\tau}_i}{\sum_{k}^{N}a^{\tau}_k} \end{equation}\markdownRendererInterblockSeparator
{}$\tau$ here shapes the contrast between well-supported and poorly supported items: exponentiating a large activation and a small activation by a large value of $\tau$ widens the difference between those activations, making recall of the most activated item even more likely. Small values of $\tau$ can alternatively driven recall likelihoods of differentially activated items toward one another.\markdownRendererInterblockSeparator
{}If an item is recalled, then that item is reactivated on $F$, and its contextual associations retrieved for integration into context again according to:\markdownRendererInterblockSeparator
{}\begin{equation} \label{eq:13} c^{IN}_{i} = M^{FC}f_{i} \end{equation}\markdownRendererInterblockSeparator
{}Context is updated again based on this input (using $\beta_{rec}$ instead of $\beta_{enc}$) and used to cue a successive recall attempt. This process continues until recall stops.\relax