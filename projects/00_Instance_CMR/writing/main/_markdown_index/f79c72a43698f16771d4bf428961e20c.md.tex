Models of memory encoding and retrieval typically include some mechanism for abstraction, a process by which information is extracted across individual experiences \markdownRendererCite{1}+{}{}{yee2019abstraction}. Abstraction is understood to involve identifying and highlighting features common across experiential episodes while disregarding or suppressing more idiosyncratic properties. This capacity for selective generalization across recurrent features of past experience is central to how memory systems retrieve relevant information even when prompted by novel cues.\markdownRendererInterblockSeparator
{}Depending on how they characterize the process abstraction, memory models can be categorized as either prototype- or instance-based. Prototype-based model conceptualize abstraction as a process enacted during encoding. Rather than stored as discrete traces, new experiences are conceived as updating prototype representations to reflect common features across past experience. Connectionist models such as the multilayer perceptron are typically examples of prototype-based models \markdownRendererCite{1}+{}{}{jamieson2018instance}; instead of being stored as separate records in memory, learning examples presented to a connectionist model each update a prototypical pattern of weights that eventually map memory probes to responses.\relax