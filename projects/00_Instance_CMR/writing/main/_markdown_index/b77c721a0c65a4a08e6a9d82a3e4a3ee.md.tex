\markdownRendererHeadingOne{Analysis Approach}\markdownRendererInterblockSeparator
{}Our simulation analyses were designed to determine whether instance-based and prototype-based instantiations of CMR can similarly account for behavioral performance in the free recall task. This includes key benchmark phenomena such as the temporal contiguity and serial position effects, as well as for the overall sequence of responses generated by participants. We used a likelihood-based model comparison technique introduced by \markdownRendererTextCite{1}+{}{}{kragel2015neural} that assesses model variants based on how accurately they can predict the specific sequence in which items were recalled. For each model, we related this technique with an optimization technique called differential evolution \markdownRendererCite{1}+{}{}{storn1997differential} to search for the parameter configuration that maximize the likelihood of the considered data. Likelihoods assigned to datasets by models and their respective optimized parameters in turn support comparison of their effectiveness accounting for the recall sequences exhibited by participants in the data. Visualization of datasets compared to those of simulation outputs given these models with these parameters similarly help compare how well models realize temporal contiguity and serial position effects.\relax