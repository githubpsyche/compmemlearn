Retrieved context theories explain memory search in terms of interactions between between two representations across experience: one of temporal context (a context layer, $C$) and another of features of studied items (an item layer, $F$). While this paper introduces an instance-based account of these interactions, we specify a variant of the original prototype-based context maintenance and retrieval (CMR model) \markdownRendererCite{1}+{}{}{polyn2009context} to support comparison against this account. The instance-based model we emphasize tracks the history of interactions between context and item features by storing a discrete record of each experience in memory for later inspection. In contrast, prototypical CMR maintains a simplified neural network whose connection weights accumulate a collapsed average representation of context and item interactions across experience. Table ~\ref{table:cmr_parameters} provides an overview of the parameters and structures that determine model behavior.\markdownRendererInterblockSeparator
{}\markdownRendererHeadingTwo{Initial State}\markdownRendererInterblockSeparator
{}Associative connections built within this network are represented by matrices $M^{FC}$ and $M^{CF}$.\markdownRendererInterblockSeparator
{}To summarize pre-experimental associations built between relevant item features and possible contextual states, we initialize $M^{FC}$ accordingly\markdownRendererInterblockSeparator
{}$$M^{FC}_{pre(ij)} = \begin{cases} \begin{alignedat}{2} 1 - \gamma \text{, if } i=j \ 0 \text{, if } i \neq j\ \end{alignedat} \end{cases}$$\markdownRendererInterblockSeparator
{}This connects each unit on $F$ to a unique unit on $C$. Used this way, $\gamma$ controls the relative contribution of pre-experimentally acquired associations to the course of retrieval compared to experimentally acquired associations. Correspondingly, context-to-feature associations tracked by $M^{CF}$ are set according to:\markdownRendererInterblockSeparator
{}$$M^{CF}_{pre(ij)} = \begin{cases} \begin{alignedat}{2} 1 - \delta \text{, if } i=j \ \alpha \text{, if } i \neq j\ \end{alignedat} \end{cases}$$\markdownRendererInterblockSeparator
{}Like $\gamma$ with respect to $M^{FC}$, the $\delta$ parameter controls the contribution of pre-experimental context-to-feature associations relative to experimentally acquired ones. Since context-to-feature associations organizes the competition of items for retrieval, the $\alpha$ parameter further causes items to support one another in retrieval in a uniform way.\markdownRendererInterblockSeparator
{}Context is initialized with a state orthogonal to any of those pre-experimentally associated with an relevant item feature. Feature representations corresponding to items are also assumed to be orthonormal with respect to one another such that each unit on $F$ corresponds to one item.\markdownRendererInterblockSeparator
{}\markdownRendererHeadingTwo{Encoding Phase}\markdownRendererInterblockSeparator
{}Whenever an item $i$ is presented for study, its corresponding feature representation $f_i$ is activated on $F$ and its contextual associations encoded into $M^{FC}$ are retrieved, altering the current state of context $C$.\markdownRendererInterblockSeparator
{}The input to context is determined by:\markdownRendererInterblockSeparator
{}$$c^{IN}\markdownRendererEmphasis{{i} = M^{FC}f}{i}$$\markdownRendererInterblockSeparator
{}and normalized to have length 1. Context is updated based on this input according to:\markdownRendererInterblockSeparator
{}$$c\markdownRendererEmphasis{i = \rho}ic\markdownRendererEmphasis{{i-1} + \beta}{enc} c_{i}^{IN}$$\markdownRendererInterblockSeparator
{}with $\beta$ (for encoding we use $\beta\markdownRendererEmphasis{enc$) shaping the rate of contextual drift with each new experience, and $\rho$ enforces the length of $c}i$ to 1 according to:\markdownRendererInterblockSeparator
{}$$\rho\markdownRendererEmphasis{i = \sqrt{1 + \beta^2\left[\left(c\markdownRendererEmphasis{{i-1} \cdot c^{IN}}i\right)^2 - 1\right]} - \beta\left(c}{i-1} \cdot c^{IN}_i\right)$$\markdownRendererInterblockSeparator
{}Associations between each $c\markdownRendererEmphasis{i$ and $f}i$ are built through Hebbian learning:\markdownRendererInterblockSeparator
{}$$\Delta M^{FC}\markdownRendererEmphasis{{exp} = \gamma c}i f^{'}_i$$\markdownRendererInterblockSeparator
{}and\markdownRendererInterblockSeparator
{}$$\Delta M^{CF}\markdownRendererEmphasis{{exp} = \phi}i f\markdownRendererEmphasis{i c^{'}}i$$\markdownRendererInterblockSeparator
{}where $\phi_i$ enforces a primacy effect, scales the amount of learning based on the serial position of the studied item according to\markdownRendererInterblockSeparator
{}$$\phi\markdownRendererEmphasis{i = \phi}se^{-\phi_d(i-1)} + 1$$\markdownRendererInterblockSeparator
{}This function decays over time, such that $\phi\markdownRendererEmphasis{se$ modulates the strength of primacy while $\phi}d$ modulates the rate of decay.\relax