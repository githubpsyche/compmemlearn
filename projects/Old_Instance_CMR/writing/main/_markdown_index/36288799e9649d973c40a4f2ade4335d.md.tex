While previous analyses evince that our prototype-based and instance-based implementations of CMR equivalently account for free recall performance when each unique item is presented just once during study, there is reason to suspect that the models might diverge when it comes to accounting for the effect of item repetition on later free recall.\markdownRendererInterblockSeparator
{}Previous work \markdownRendererCite{1}+{}{}{siegel2014retrieved} has related CMR to two broad accounts of how item repetition influences memory and in particular drives the spacing effect, a monotonic relationship between recall probability and the size of the lag between item repetitions in a study list. Under the contextual-variability account \markdownRendererCite{1}+{}{}{anderson1972recognition}, each time an item is studied, it's associated in memory with the current state of a slowly drifting contextual representation. Depending on how spaced apart two presentations of an item might be, the contextual states they are associated with might either be very similar or very distinct. Later, participants use the current state of their contextual representation to probe their memories and retrieve items during free recall. When an item has been associated with diverse contextual states, it can correspondingly be retrieved using diverse possible cues. In this way, the improvements in recall we gain from spacing presentations of an item are explained in terms of variation in the range of possible cues that can trigger recall of that item. A study-phase retrieval account of the spacing effect alternatively emphasizes the consequences of studying a successively presented item. According to the account, when this happens we retrieve memories of the repeated item's earlier occurrences and their associated contexts. When this happens, it's proposed that retrieved information is memorally associated with information corresponding to the current presentation context.\markdownRendererInterblockSeparator
{}Analyses of our instance-based implementation of CMR so far suggest it realizes these mechanisms similarly to the original prototype-based CMR. A potentially more relevant distinction between the models might instead turn on differences in how records of past experience are integrated for retrieval. InstanceCMR, like MINERVA 2, has the option to apply its nonlinear activation scaling parameter $\tau$ to activations of individual traces - that is, before integration into a unitary vector tracking retrieval support. However, CMR does not access trace activations and applies $\tau$ to the integrated echo representation result.\markdownRendererInterblockSeparator
{}This distinction between instance-based and prototype-based architectures has been marshalled to explain model differences in other research contexts \markdownRendererCite{1}+{e.g.,}{}{jamieson2018instance}. In this context, however, the different between applying $\tau$ to trace activations or echo content is between enforcing quasi-linear or quasi-exponential effect of item repetition on subsequent recall probability. Suppose a constant sensitivity parameter $\tau$ and that two distinct experiences each contributed a support of $c$ for a given feature unit in the current recall. Under trace-based sensitivity scaling, the retrieval support for that feature unit would be $c^{\tau} + c^{\tau}$. But under echo-based sensitivity scaling, support would be ${(c + c)}^{\tau}$, a much larger quantity.\markdownRendererInterblockSeparator
{}Another way to illustrate this architectural difference is by simulation. We can have our prototype-based and each variant of our instance-based implementation of CMR simulate a traditional list-learning experiment with study of 20 unique items in order. Then, we can simulate repeated study of an arbitrary item in that list and measure the effect on the probability of retrieving that item through context-based recall for each successive repetition. ~\ref{fig:repeffect} plots the result of of this simulation over 1000 experiments for 50 item repetitions using PrototypeCMR and InstanceCMRand model parameters fitted using data from \markdownRendererTextCite{1}+{}{}{murdock1970interresponse}. Model fitting over a different dataset might obviate these observed differences; however these simulations raise the possibility that with increasing item repetitions, prototype-based and instance-based implementations of CMR might support different predictions about the influence of item repetition on later recall probability, motivating further investigation.\relax