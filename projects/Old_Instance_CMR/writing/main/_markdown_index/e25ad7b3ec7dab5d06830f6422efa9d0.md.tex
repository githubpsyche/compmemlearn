Retrieved context theories explain memory search in terms of interactions between between two representations across experience: one of temporal context (a context layer, $C$) and another of features of studied items (an item layer, $F$). While this paper introduces an instance-based account of these interactions, we specify a variant of the original prototype-based context maintenance and retrieval (CMR model) \markdownRendererCite{1}+{}{}{polyn2009context} to support comparison against this account. The instance-based model we emphasize tracks the history of interactions between context and item features by storing a discrete record of each experience in memory for later inspection. In contrast, prototypical CMR maintains a simplified neural network whose connection weights accumulate a collapsed average representation of context and item interactions across experience. Table provides an overview of the parameters and structures that determine model behavior.\markdownRendererInterblockSeparator
{}::: {#table:cmr\markdownRendererEmphasis{parameters} Structure Type Symbol Name Description ----------------------- ------------------- ------------------------- ------------------------------------------------------------- Architecture\markdownRendererLineBreak
{}$C$ temporal context A recency-weighted average of encoded items $F$ item features Current pattern of item feature unit activations $M^{FC}$ encoded feature-to-context associations $M^{CF}$ encoded context-to-feature associations Context Updating\markdownRendererLineBreak
{}${\beta}}{enc}$ encoding drift rate Rate of context drift during item encoding ${\beta}\markdownRendererEmphasis{{start}$ start drift rate Amount of start-list context retrieved at start of recall ${\beta}}{rec}$ recall drift rate Rate of context drift during recall Associative Structure\markdownRendererLineBreak
{}${\alpha}$ shared support Amount of support items initially have for one another ${\delta}$ item support Initial pre-experimental contextual self-associations ${\gamma}$ learning rate Amount of experimental context retrieved by a recalled item ${\phi}\markdownRendererEmphasis{{s}$ primacy scale Scaling of primacy gradient on trace activations ${\phi}}{d}$ primacy decay Rate of decay of primacy gradient Retrieval Dynamics\markdownRendererLineBreak
{}${\tau}$ choice sensitivity Exponential weighting of similarity-driven activation ${\theta}\markdownRendererEmphasis{{s}$ stop probability scale Scaling of the stop probability over output position ${\theta}}{r}$ stop probability growth Rate of increase in stop probability over output position\markdownRendererInterblockSeparator
{}: Parameters and structures specifying CMR :::\relax