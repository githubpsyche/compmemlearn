Retrieved context theories explain memory search in terms of interactions between between two representations across experience: one of temporal context (a context layer, $C$) and another of features of studied items (an item layer, $F$). While this paper introduces an instance-based account of these interactions, we specify a variant of the original prototype-based context maintenance and retrieval (CMR model) \markdownRendererCite{1}+{}{}{polyn2009context} to support comparison against this account. The instance-based model we emphasize tracks the history of interactions between context and item features by storing a discrete record of each experience in memory for later inspection. In contrast, prototypical CMR maintains a simplified neural network whose connection weights accumulate a collapsed average representation of context and item interactions across experience. Table provides an overview of the parameters and structures that determine model behavior.\markdownRendererInterblockSeparator
{}\markdownRendererTable{Parameters and structures specifying CMR :::}{3}{4}{dddd}%
{{Structure Type}%
{Symbol}%
{Name}%
{Description}%
}%
{{Architecture}%
{}%
{}%
{}%
}%
{{$C$}%
{temporal context}%
{A recency-weighted average of encoded items}%
{}%
}%
\relax