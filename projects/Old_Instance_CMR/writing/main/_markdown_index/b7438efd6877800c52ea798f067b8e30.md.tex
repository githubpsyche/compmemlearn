\markdownRendererHeadingOne{The Prototype-Based Account of Context Maintenance and Retrieval}\markdownRendererInterblockSeparator
{}Retrieved context theories explain memory search in terms of interactions between between two representations across experience: one of temporal context (a context layer, $C$) and another of features of studied items (an item layer, $F$). While this paper introduces an instance-based account of these interactions, we here specify a variant of the original prototype-based context maintenance and retrieval (CMR model) \markdownRendererCite{1}+{}{}{polyn2009context} to support comparison against this account. The instance-based model we emphasize tracks the history of interactions between context and item features by storing a discrete record of each experience in memory for later inspection. In contrast, prototypeCMR maintains a simplified neural network whose connection weights accumulate a center of tendency representation reflecting context and item interactions across experience. \markdownRendererInterblockSeparator
{}\markdownRendererTable{Parameters and structures specifying CMR}{20}{4}{llll}%
{{Structure Type}%
{Symbol}%
{Name}%
{Description}%
}%
{{Architecture}%
{}%
{}%
{}%
}%
{{}%
{$C$}%
{temporal context}%
{A recency-weighted average of encoded items}%
}%
{{}%
{$F$}%
{item features}%
{Current pattern of item feature unit activations}%
}%
{{}%
{$M^{FC}$}%
{}%
{encoded feature-to-context associations}%
}%
{{}%
{$M^{CF}$}%
{}%
{encoded context-to-feature associations}%
}%
{{Context Updating}%
{}%
{}%
{}%
}%
{{}%
{${\beta}_{enc}$}%
{encoding drift rate}%
{Rate of context drift during item encoding}%
}%
{{}%
{${\beta}_{start}$}%
{start drift rate}%
{Amount of start-list context retrieved at start of recall}%
}%
{{}%
{${\beta}_{rec}$}%
{recall drift rate}%
{Rate of context drift during recall}%
}%
{{Associative Structure}%
{}%
{}%
{}%
}%
{{}%
{${\alpha}$}%
{shared support}%
{Amount of support items initially have for one another}%
}%
{{}%
{${\delta}$}%
{item support}%
{Initial pre-experimental contextual self-associations}%
}%
{{}%
{${\gamma}$}%
{learning rate}%
{Amount of experimental context retrieved by a recalled item}%
}%
{{}%
{${\phi}_{s}$}%
{primacy scale}%
{Scaling of primacy gradient on trace activations}%
}%
{{}%
{${\phi}_{d}$}%
{primacy decay}%
{Rate of decay of primacy gradient}%
}%
{{Retrieval Dynamics}%
{}%
{}%
{}%
}%
{{}%
{${\tau}$}%
{choice sensitivity}%
{Exponential weighting of similarity-driven activation}%
}%
{{}%
{${\theta}_{s}$}%
{stop probability scale}%
{Scaling of the stop probability over output position}%
}%
{{}%
{${\theta}_{r}$}%
{stop probability growth}%
{Rate of increase in stop probability over output position}%
}%
\markdownRendererInterblockSeparator
{}\markdownRendererHeadingTwo{Initial State}\markdownRendererInterblockSeparator
{}Associative connections built within prototypeCMR are represented by matrices $M^{FC}$ and $M^{CF}$.\markdownRendererInterblockSeparator
{}To summarize pre-experimental associations built between relevant item features and possible contextual states, we initialize $M^{FC}$ according to:\markdownRendererInterblockSeparator
{}\begin{equation} \label{eq:1} M^{FC}_{pre(ij)} = \begin{cases} \begin{alignedat}{2} 1 - \gamma \text{, if } i=j \\ 0 \text{, if } i \neq j \end{alignedat} \end{cases} \end{equation}\markdownRendererInterblockSeparator
{}This connects each unit on $F$ to a unique unit on $C$. Used this way, $\gamma$ controls the relative contribution of pre-experimentally acquired associations to the course of retrieval compared to experimentally acquired associations. Correspondingly, context-to-feature associations tracked by $M^{CF}$ are set according to:\markdownRendererInterblockSeparator
{}\begin{equation} \label{eq:2} M^{CF}_{pre(ij)} = \begin{cases} \begin{alignedat}{2} 1 - \delta \text{, if } i=j \\ \alpha \text{, if } i \neq j \end{alignedat} \end{cases} \end{equation}\markdownRendererInterblockSeparator
{}Like $\gamma$ with respect to $M^{FC}$, the $\delta$ parameter controls the contribution of pre-experimental context-to-feature associations relative to experimentally acquired ones. Since context-to-feature associations organizes the competition of items for retrieval, the $\alpha$ parameter specifies a uniform baseline extent to which items support one another in that competition.\markdownRendererInterblockSeparator
{}Context is initialized with a state orthogonal to any of those pre-experimentally associated with an relevant item feature. Feature representations corresponding to items are also assumed to be orthonormal with respect to one another such that each unit on $F$ corresponds to one item.\markdownRendererInterblockSeparator
{}\markdownRendererHeadingTwo{Encoding Phase}\markdownRendererInterblockSeparator
{}Whenever an item $i$ is presented for study, its corresponding feature representation $f_i$ is activated on $F$ and its contextual associations encoded into $M^{FC}$ are retrieved, altering the current state of context $C$.\markdownRendererInterblockSeparator
{}The input to context is determined by:\markdownRendererInterblockSeparator
{}\begin{equation} \label{eq:3} c^{IN}_{i} = M^{FC} \end{equation}\markdownRendererInterblockSeparator
{}```python\markdownRendererInterblockSeparator
{}```\relax