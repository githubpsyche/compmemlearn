Recall for narratives, if split into idea units -- "meaningful chunks of information that convey a piece of the narrative" -- that are numbered according to chronological order, can be examined using analytic techniques developed for free and serial list recall tasks. This framework enables direct comparison between ideas, assumptions, and models applied to understand how people remember sequences such as word lists and those used to understand memory for narrative texts. To support analysis of narrative recall this way, we considered a dataset collected, preprocessed, and presented by \citet{cutler2019narrative}. In corresponding experiments, research participants read 6 distinct short stories. Upon reading a story, participants performed immediate free recall of the narrative twice. Three weeks later, participants performed free recall of each narrative again. Each recall period was limited to five minutes. Following data collection, a pair of research assistants in the Brown-Schmidt laboratory were each instructed to independently split stories and participant responses into idea units as defined above, and to identify correspondences between idea units in participant responses and corresponding studied stories reflecting recall. Following this initial preprocessing, research assistants then compared and discussed their results and recorded consensus decisions regarding the segmentation and correspondence of idea units across the dataset. Further analysis focused on the sequences of story idea units recalled by participants on each trial as tracked by these researchers.